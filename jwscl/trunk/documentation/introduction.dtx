
@@introduction.txt
<title Introduction>

<image jwscllogotxtnew01-small>

<b>JEDI Windows Security Library BETA</b>

The suite is a bunch of classes that encapsulated MS Windows security functions
in object oriented way.

The goal of this project is to make security programming easier to Delphi
programmers. This includes that the programmer must not allocate Windows API
memory and also free it. Furthermore she can use simple Delphi String (Ansi- and
WideString) in all function that needs string manipulation. The memory
allocation and deallocation for these things is done internally with a content
check. Of course it is possible to retrieve WinAPI structures easily to use in
other WinAPI calls (e.g. <link TJwSecurityDescriptor>.<link TJwSecurityDescriptor.Create_SD@Cardinal@boolean, Create_SD>).
The caller also has to free it using a counter part function (e.g. <link TJwSecurityDescriptor>.<link TJwSecurityDescriptor.Free_SD@PSecurityDescriptor, Free_SD>).

Also API functions are wrapped together in classes to diminish initialization
that consumes a great amount of time which (s)he could use for better things.
All the programmer has to do is to free the class instance after she has used
it.

<b>However it is not the goal to teach Delphi programmers how to program in
windows security or create secure windows applications</b>. The programmer has
to understand the security of windows by herself. To use these classes she must
know what tokens, principals, secure objects, sessions and so on are.

On the other hand if you want to test the software by creating good test
scenarios (module, function, class, integration testing) you are welcome. With
good tests I mean tests that are reproducible, have a high function cover and
uses DUnit.

<b>This library has beta status but it does not mean that there are major
problems to be expected. It means that not all features are tested well and
radically changes can break source compatibility (e.g. changing names - seldom).</b>

The following table shows supported Delphi versions:
<table 79%, 39c%, 71c%>
Library                   Supported Delphi versions
------------------------  ----------------------------------------------------
JEDI Windows API Header   Delphi 5 to Delphi 2009
 Conversions               
JEDI Windows Security     Delphi 7 to Delphi 2009<p />Currently there is no
 Code Library              intention to support Delphi 5,6 because there isn't
                           just enough man power.
</table>

For information about <b>license facts</b> see <link conclusion.txt, License page>.

Do not forget to send bugs, comments or greets to the authors.

Christian Wimmer – mail[at]delphi-jedi[dot]net

Stuttgart, Germany



Contributors (in alphabetical order)
  * Philip Dittmann
  * Danila Galimov
  * Binh Ly (ComLib.pas)
  * Remko Weijnen
In Memory Of Robert Marquardt



Visit Source Forge Project
  * <extlink http://sourceforge.net/projects/jedi-apilib>Sourceforge</extlink>
  * <extlink http://blog.delphi-jedi.net>Webblog</extlink>

* Contact options at a glance *
<table 50c%>
###################  ################################################################################################################################################################
Mail contact         mail[at]delphi-jedi[dot]net
Mailing list         Subscribe to the mailing lists at sourceforge:<p /><extlink http://sourceforge.net/mail/?group_id=121894>http://sourceforge.net/mail/?group_id=121894</extlink>
Newsgroups           1. <extlink news://forums.talkto.net:119/jedi.apiconversion>news://forums.talkto.net:119/jedi.apiconversion</extlink>
                      2. <extlink news://forums.talkto.net:119/jedi.general>news://forums.talkto.net:119/jedi.general</extlink>
JEDI Issue           <extlink http://homepages.codegear.com/jedi/issuetracker>http://homepages.codegear.com/jedi/issuetracker</extlink>
 Tracker<p />Select   
 "API &amp; WSC       
 Library" in the      
 upper right          
 corner and           
 switch.              
</table>
